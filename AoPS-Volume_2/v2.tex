\documentclass{scrartcl}
\usepackage[sexy]{evan}
\usepackage{amsmath}

\newtheorem{a_problem}{Problem}

\title{AoPS Volume 2 Solutions}
\author{Mckinley Xie}

\begin{document}
\maketitle
\tableofcontents

\setcounter{section}{22}
\section{Number Theory}
\begin{exercise}
	Show that if $a_1 \mid (a,b)$, then $\frac{b}{(a,b)} \mid \frac{b}{a_1}$
	\begin{soln}
		$(a,b)$ is a multiple of $a_1$, so it's pretty clear.
	\end{soln}
\end{exercise}

\begin{exercise}
	Compare our $2 \equiv 20$ (mod 6) example to equation (23.2)
	\begin{soln}
		It just states $1 \equiv 10$ (mod 3)
	\end{soln}
\end{exercise}

\begin{exercise}
	Divide out the common factors in the following congurences:
	\begin{enumerate}
	\item $6a \equiv 6b$ (mod 20)
	\item $23 \equiv 138$ (mod 5)
	\item $12 \equiv 30$ (mod 9)
	\end{enumerate}
	\begin{soln} Just regurgitate what we learned in the lesson:
		\begin{enumerate}
			\item $a \equiv b$ (mod 10)
			\item $1 \equiv 6$ (mod 5)
			\item $2 \equiv 5$ (mod 3)
		\end{enumerate}
	\end{soln}
\end{exercise}

\begin{exercise}
	Solve the congurences.
	\begin{enumerate}
		\item $1235x + 45 \equiv 9090$ (mod 24)
		\item $1235x + 45 \equiv 9090$ (mod 11)
		\item $1235x + 45 \equiv 9087$ (mod 11)
		\item $1232x + 45 \equiv 9090$ (mod 24)
	\end{enumerate}
	\begin{soln} Nothing special to see here. 
		\begin{enumerate}

			\item 
				\begin{align*}
					1235x + 45 &\equiv 9090 \pmod{24} \\
					1235x &\equiv 9045 \\
					247x &\equiv 1809 \\
					7x &\equiv 9 \\
					7x &\equiv 105 \\ 
					x &\equiv 15 \pmod{24}
				\end{align*}
			\item 
				\begin{align*}
					1235x + 45 &\equiv 9090 \pmod{11} \\
					247x &\equiv 1809  \\
					5x &\equiv 5  \\
					x &\equiv 1 \pmod{11}
				\end{align*}
			\item 
				\begin{align*}
					1235x + 45 &\equiv 9087 \pmod{11} \\
					3x + 1 &\equiv 1 \\
					x &\equiv 0 \pmod{11}
				\end{align*}
			\item 
				\begin{align*}
					1232x + 45 &\equiv 9090 \pmod{24} \\
					1232x &\equiv 9045 \\
					8x &\equiv 21 \pmod{24}
				\end{align*}
				So there is no solution.
			\begin{remark*}
				You can also just take everything mod 2 to get $1 \equiv 0 \pmod{2}$
			\end{remark*}
		\end{enumerate}
	\end{soln}
\end{exercise}

\begin{exercise}
	Solve simultaneously the three congruences $3x \equiv 4 \pmod{7}$, $4x \equiv 5 \pmod{8}$, and $5x \equiv 6 \pmod {9}$
	\begin{soln}
		In the first equation, $x \equiv 6 \pmod{7}$. \\
		In the second, $4x \equiv 5 \pmod{8}$ is impossible.\\
		Oops.
	\end{soln}
\end{exercise}

\begin{definition*}
	A \textbf{quadratic residue}$\pmod{m}$ is a number $n$ such that $\exists i$ such that $i^2 \equiv n \pmod{m}$
\end{definition*}

\begin{exercise}
Find all quadratic residues in mod 7, 8, and 9.
	\begin{soln}
		We're lazy so we'll just do mod 7.
		Well, we have $0^2, 1^2, 2^2$, and $3^2 \pmod{7}$, so \fbox{0, 1, 2, and 4}.
	\end{soln}
\end{exercise}

\begin{exercise}
	What is the most quadratic residues there can be $\pmod{n}$ for $n=2m+1$?
	\begin{soln}
		Because every number $k$ has the same (residue?) as $n - k$, our answer is $m+1 + 1 = \boxed{m+2}$ (since 0 goes to itself.)
	\end{soln}
\end{exercise}

\setcounter{exercise}{9}

\begin{exercise}
	Write down and expand the product for $n=28$
	\begin{soln}
		We have $(1 + 2 + 4)(1 + 7) = 56$
	\end{soln}
\end{exercise}

\begin{exercise}
	Why does this work?
	\begin{soln}
		It's like a genfunc!
	\end{soln}
\end{exercise}

\begin{exercise}
	Make the product simpler with sum of geometric series
	\begin{soln}
		\[\prod_{i = 0}^k \frac{p_i^{e_i + 1}-1}{p_i - 1}\]
		Honestly this isn't much simpler.
	\end{soln}
\end{exercise}

\begin{exercise}
	These are pretty obviously correct, but I'm not particularly fond of either since they're not very helpful.
\end{exercise}

\setcounter{exercise}{14}

\begin{exercise}
	Show that any perfect number of the form
	\[2^k\left(2^{k+1}-1\right)\]
	is perfect if $(2^{k+1} - 1)$ is prime.

	This was discovered by Euclid, and \emph{all known perfect numbers have this form.} (In particular, no odd perfect numbers have ever been found.)
	\begin{soln}
		By our handy-dandy formula,
		the sum of divisors is equal to
		\[(2^{k+1} - 1)(2^{k+1} - 1 + 1)\]
		Which is twice our original number, and we're done.
	\end{soln}
\end{exercise}

\setcounter{exercise}{18}

\begin{exercise}
	Find $6^{1000} \pmod{23}$
	\begin{soln}
		$6^{22} \equiv 1 \pmod{23}$, so $6^{990} \equiv 1 \pmod{23}$. 
		We then bash out $6^{10} \pmod{23}$ and find that it's equal to \boxed{4}
	\end{soln}
\end{exercise}

\begin{exercise}
	Find all possible periods a number can have $\pmod{23}$
	\begin{soln}
		Well, $a^{22}$ must be 1, so the period length must be 1, 2, 11, or 22.
	\end{soln}
\end{exercise}
\begin{definition*}
	A \textbf{primitive root $\pmod{p}$} is a number $g$ with period $p-1$.
\end{definition*}
\begin{exercise}
	Let the divisors of $p-1$ be $d_1, d_2,\dots$ Prove that if we have a primitive root $g \pmod{p}$, then for each $d_i$ there is an element with period $d_i$.
	\begin{soln}
		If $g$ has period $p-1$, then $g^{\frac{p-1}{d_i}}$ has period $d_i$.
	\end{soln}
\end{exercise}

\setcounter{exercise}{23}

\begin{exercise} It's just $p^{k-1}$ \end{exercise}

\setcounter{exercise}{26}

\begin{exercise} We get $(m-1)! \equiv 0 \pmod{a}$ \end{exercise}

\begin{exercise}
	Look at the proof of Wilson's theorem
	\begin{proof}
		Consider some primitive root $g \pmod{p}$. By the definition of a primitive root, $\{g^k \mid k \in [p-1] \} = [p-1]$. So
		\[(p-1)! \equiv g^{\frac{p(p-1)}{2}} \pmod{p}\]
	But because $g^p \equiv g\left(g^{p-1}\right) \equiv g \pmod{p}$, $g^{\frac{p(p-1)}{2}} \equiv {\left(g^p\right)}^{\frac{p-1}{2}} \equiv g^{\frac{p-1}{2}} \pmod{p}$.
		Let $t = g^{\frac{p-1}{2}}$. Note that
		\[t^2 = g^{p-1} \equiv 1 \pmod{p}\]
		So $t^2 - 1 \equiv 0 \pmod{p}$, and $t \equiv \pm 1 \pmod{p}$ \\
		But because $t = g^{\frac{p-1}{2}}$, we can't have $t \equiv 1$ because the period of $g$ is $p-1$ (and $\frac{p-1}{2} < p-1$), so $t \equiv -1$, and $(p-1)! \equiv -1 \pmod{p}$
	\end{proof}
\end{exercise}

\setcounter{a_problem}{372}
\begin{a_problem}
	Show that for all prime numbers $p$ greater than 3, 24 divides $p^2 - 1$ evenly.
	\begin{proof}
		$p^2 - 1 = (p+1)(p-1)$. If $p$ is prime then $p \equiv 1$ or $p \equiv 3 \pmod{4}$ so $(p+1)(p-1)$ is divisible by 8. Similarly, $p \equiv 1$ or $2 \pmod{3}$ so one of the factors of $p^2 - 1$ is divisible by 3, and we're done.
	\end{proof}
\end{a_problem}

\begin{a_problem}
	Given that $n-4$ is divisible by 5, list which of the following are also divisible by 5:
	\[n^2 - 1, n^2 - 4, n^2 - 16, n + 4, n^4 - 1\]
	(Mandelbrot \#3)
	\begin{soln}
		Well, just take everything mod $5$. \\
		$n \equiv 1 \pmod{5}$ so $n^2 - 1$, $n^2 - 16$, $n+4$, and $n^4 - 1$ are all divisible by 5.
	\end{soln}
\end{a_problem}

\begin{a_problem}
	If the same number $r$ is the remainder when each of the numbers 1059, 1417, and 2312 is divided by $d$, where $d$ is an integer greater than one, find $d-r$.
	\begin{soln}
		Two pairwise differences between these numbers are 358 and 895. These are both divisible by 179, so $d$ is 179. Plugging in, $r$ is 164, so our answer is \boxed{15}
	\end{soln}
\end{a_problem}

\begin{a_problem}
	Find the sum of all $x$, $1 \leq x \leq 100$, such that 7 divides $x^2 + 15x + 1$. (Mandelbrot \#3)
	\begin{soln}
		The condition is equivalent to $x^2 + x + 1$.
		We just test all numbers $\pmod{7}$.
		1 doesn't work, 2 does, 3 doesn't, 4 does, 5 doesn't, and 6 doesn't. \\
		We can just bash this out.
	\end{soln}
\end{a_problem}

\begin{a_problem}
	Find the largest integer divisor of $n^5 - n$
	\begin{soln}
		This expression is equal to $n(n^2+1)(n+1)(n-1)$. \\
		Note that if $n=2$ then our expression is equal to $30$, so our largest integer must be a divisor of $30$. 
		\begin{claim*}
		30 always divides $n^5 - n$ (for $n \in \mathbb{Z}$). 
		\end{claim*}
		Clearly one of $n$ and $n-1$ are divisible by 2, so our expression is always divisible by 2. Similarly, one of $n-1$, $n$, and $n+1$ are divisible by 3. \\
		Now, suppose that this expression is not divisible by 5 for some $n$.
		That implies that each of $n-1, n, n+1 \not\equiv 0 \pmod{5}$.
		That means that $n$ must be either 2 or 3 $\pmod{5}$.
		But $2^2 + 1 \equiv 3^2 + 1 \equiv 5 \pmod{5}$, which is a contradiction since $n$ is not divisible by 5. \\
		Because $2,3,5$ all divide $n^5 - n$ then $\mid n^5 - n$ and we are done.
	\end{soln}
	\begin{remark}
		Another solution could be to use induction and show that 
		\[30 \mid ({(n+1)}^5 - (n+1)) - (n^5 - n)\]
		However this is terribly messy and sad.
	\end{remark}
\end{a_problem}

\begin{a_problem}
	What is the units digit of $7^{\left(7^7\right)}$?
	\begin{soln}
		By Fermat's theorem we know that $7^4 \equiv 1 \pmod{10}$. \\
		Now we need to find $7^7 \pmod{4}$. \\
		Well, we know that $7^2 \equiv 1 \pmod 4$ so $7^7 \equiv 3 \pmod4$ \\
		So our answer is \boxed{3}.
	\end{soln}
\end{a_problem}

\begin{a_problem}
	What is the size of the largest subset $S$ of [50] such that no pair of distinct elements of $S$ has a sum divisible by 7?
	\begin{soln}
	Clearly if we have a number $a \in S$, we cannot have $b \equiv -a$, where $b \in S$. \\
	So we just take all $x \equiv 1,2,3 \pmod7$. \\
	We can also throw in a $7$, so our answer is $22 + 1 = \boxed{23}$
	\end{soln}
\end{a_problem}

\setcounter{a_problem}{380}
\begin{a_problem}
	For any integer $n$ greater than 1, how many prime numbers are there greater than $n! + 1$ and less than $n! + n$?
	\begin{soln}
		$k \mid n! + k$ for $k \leq n$, so our answer is \boxed{0}.
	\end{soln}
\end{a_problem}
\begin{a_problem}
	Find the last three digits of $9^{105}$.
	\begin{soln}
		This is equivalent to $3^{210}$. \\
		$3^{400} \equiv 1 \pmod{1000}$ so $3^{200} \equiv -1 \pmod{1000}$. \\
		We then bash out the last three digits of $3^{10}$, which are 049, so our answer is \boxed{951}
		
	\end{soln}
\end{a_problem}

\begin{a_problem}
	What is the least possible value of $n$ such that
	\[\sqrt{\frac31 \cdot \frac42 \cdot \frac53 \cdots \frac{n+2}n}\]
	is an integer?
	\begin{soln}
		This just cancels out to
		\[\sqrt{\frac{(n+2) \cdot (n+1)}{2}}\]
		Assuming $n \in \mathbb{Z}$, then one factor is a perfect square while the other is twice a perfect square. We test small perfect squares until we stumble upon \boxed{n=7}
	\end{soln}
\end{a_problem}

\setcounter{a_problem}{386}
\begin{a_problem}
	Let $x$ and $y$ be integers such that $2x+3y$ is a multiple of 17. Show that $9x + 5y$ must also be a multiple of 17. (USAMTS 1)
	\begin{soln}
		Note first that $-4(2x + 3y) \equiv -8x - 12y \equiv 0 \pmod{17}$. \\
		So $17x - 8x + 17y - 12y \equiv 0 \pmod{17}$, and the rest is trivial.
	\end{soln}
\end{a_problem}

\begin{a_problem}
	Note that $1990$ can be ``turned into a square'' by adding a digit on its right, and some digits on its left, i.e., $419904=648^2$. Prove that 1991 cannot be turned into a square by the same procedure; i.e., there are no digits $d,x,y, \dots$ such that $\dots yx1991d$ is a perfect square. (USAMTS 3)
	\begin{soln}
		Suppose there exists a number whose square satisfies the desired property. Let its last two digits be $a$ and $b$.
		Clearly the second-to-last digit of $20ab + b^2$ must be a 1. That implies that the tens digit of $b^2$ is odd, so $b=4$ or $b=6$. So we need to find $a$ such that $8a + 1 \equiv 1 \pmod{10}$ or $12a + 3 \equiv 1 \pmod{10}$, so we have $10a+b = $04, 46, 54, or 96.
		Uhh I dunno COME BACK HERE LATER
	\end{soln}
\end{a_problem}

\setcounter{a_problem}{392}
\begin{a_problem}
	Let $n$ be an integer. If the tens digit of $n^2$ is 7, what is the units digit of $n^2$?
	\begin{soln}
		Consider $n^2 \pmod{100}$. \\
		We want to find $10a+b$ such that $2ab + (\text{the tens digit of $b^2$}) = 7$. \\
		Clearly the tens digit of $b^2$ must be odd, so $b=4$ or $b=6$, so our units digit must be \boxed{6}.
	\end{soln}
\end{a_problem}

\begin{a_problem}
	Prove that none of the numbers $a_n = 1001001\cdots1001$ is prime, where $n=2,3,4\dots$ denotes the number of occurrences of the digit 1 in $a_n$.
	\begin{soln}
		Define a function $f(x,n)$ as
		\[f(x,n) = \sum_{k=0}^{n-1}x^k = \frac{x^{k+1}-1}{x - 1} \]
			Conveniently, $a_n = f(1000,n) = \frac{10^{3n+1} - 1}{(10 - 1)(111)}$. \\
		Now, consider only odd $n$, since $2 \mid n \implies a_2 \mid a_n$. \\
		If $n$ is odd then $a_n = \frac{1000^{n+1}-1}{999} = \frac{(1000^{\frac{n+1}{2}} + 1)(1000^{\frac{n+1}2} - 1)}{999}$, and for $n>1$ this is clearly not prime, and we're done.
	\end{soln}
	\begin{remark*}
		I hate my proof. Hopefully there's something better.
	\end{remark*}
\end{a_problem}

\begin{a_problem}
	Let $p$ be a prime number. Prove that there exists an integer $a$ such that $p \mid a^2 - a + 3 \iff \exists b$ such that $p \mid b^2 - b + 25$.
	\begin{soln}
		The problem statement is equivalent to proving that \\
		$\exists a$ such that $a(a-1) \equiv -3 \pmod{p} \iff \exists b$ such that $b(b-1) \equiv -25 \pmod{p}$ \\
		I'm not sure how to progress.
	\end{soln}
\end{a_problem}

\begin{a_problem}
	Each of the numbers $x_1, x_2,\dots x_n$ equals 1 or -1, and
	\[x_1x_2x_3x_4 + x_2x_3x_4x_5 + \cdots + x_{n-2}x_{n-1}x_{n}x_1 + x_{n-1}x_{n}x_1x_2 + x_{n}x_1x_2x_3 = 0\]
	Prove that $4 \mid n$.
	\begin{soln}
		Let term number $k$ be $y_k$. So each $y_i$ is either $1$ or $-1$, and the sum of all $y_1 + \cdots + y_n = 0 \implies 2 \mid n$. Let $n=2m$. $y_1 y_2 \cdots y_n = {\left(-1\right)}^m$, but each $x_j$ appears 4 times in this product, so $2 \mid m$ and we are done.
	\end{soln}
\end{a_problem}

\setcounter{a_problem}{397}
\begin{a_problem}
Find the positive integer $m$ such that the polynomial $p^3 + 2p + m$ divides $p^{12} - p^{11} + 3p^{10} + 11p^3 - p^2 + 23p + 10$.
	\begin{soln}
		We plug in 0 and find that $m \mid 30$. \\
		We plug in 1 and find that $m + 3 \mid 66$, so $m = 3,8,19,30,63$ \\
		So clearly $m = 3$ or $m = 30$.
		We plug in $p=2$ to get $2048 + 3072 + 88 - 4 + 46 + 30 = 5120 + 84 + 46 + 30 = 5280$, and our possible divisors are $15$ or $42$. \\
		42 doesn't work so our answer is \boxed{m=3}.
	\end{soln}
\end{a_problem}
\begin{a_problem}
	Prove that, for all positive integer pairs $(a,b)$ where $b>2, 2^b - 1$ does not evenly divide $2^a + 1$.
	\begin{soln}
		Suppose $\exists k \in \mathbb{Z}$ such that $k(2^b - 1) = 2^a + 1$. \\
		Note that $2^{b-a}\left( 2^a +1\right) = 2^b + 2^{b-a} > 2^b - 1$, so $k < 2^{b-a}$. \\
		But $\left(2^{b-a} - 1\right)\left(2^a + 1\right) = 2^b - 2^a - 1 < 2^b - 1$, so $k > 2^{b-a}-1$ \\
		This is absurd since $k$ cannot be between two integers, so we're done.

	\end{soln}
\end{a_problem}

\begin{a_problem}
Let $d$ be any positive integer not equal to 2, 5, or 13. Show that one can find distinct $(a,b)$ in the set $\{2,5,13,d\}$ such that $ab-1$ is not a perfect square.
	\begin{soln}
		Suppose there exist positive integers $x,y,z$ such that
		\begin{align*}
			x^2 &= 2d+1 \\
			y^2 &= 5d+1 \\
			z^2 &= 13d+1 \\
		\end{align*}
		Then $y^2-x^2 = 3d$, $z^2 - y^2 = 8d$, and $z^2 - x^2 = 11d$.
		Clearly $x$ is odd, so $x^2 \equiv 1 \pmod 8$, and $2d \equiv x^2 + 1 \equiv 2 \pmod 8$, so $d$ is odd. This forces $y$ and $z$ to be even, so let $y=2u$ and $z=2v$. Since $z^2 - y^2 = 8d$, we have $(v+u)(v-u) = 2d$. But because $u+v$ and $u-v$ must have the same parity, then both must be even, so $4 \mid 2d \implies 2 \mid d$, which is a contradiction.
			
	\end{soln}
\end{a_problem}

\begin{a_problem}
	Let $a$ and $b$ be integers and $n$ a positive integer. Prove that 
	\[n! \mid b^{n-1}a(a+b)(a+2b)\cdots(a+(n-1)b)\]
\end{a_problem}

\begin{a_problem}
	Prove that a positive integer is a sum of at least two consecutive positive integers if and only if it is not a power of two.
	\begin{soln}
		For some $n=2^a \cdot b$ where b is odd, we split this up into two cases. 

		Case 1. $2^a \geq \frac{b+1}{2} $ (and $b > 1$) \\
		We just take $2^a$, the $\frac{b-1}{2}$ numbers below it, and the $\frac{b-1}{2}$ numbers above it. Because $2^a > \frac{b+1}{2}$ all of these numbers are positive, and there are $b > 1$ terms in this sequence.

		Case 2. $2^a < \frac{b+1}{2} $ \\
		Consider the $2^a$ consecutive integers ending at $\frac{b-1}{2}$ and the $2^a$ consecutive integers beginning at $\frac{b+1}{2}$. The sum of these integers is obviously $2^{a}b = n$, and the minimum integer is $\frac{b+1}{2} - 2^a \geq 1$, so we're done.
	\end{soln}
\end{a_problem}

\end{document}
